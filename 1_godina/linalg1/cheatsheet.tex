\documentclass[12p,a4paper]{article}

\usepackage[serbian]{babel}
\usepackage[T2A]{fontenc} 
\usepackage[utf8]{inputenc} 
\usepackage{multicol} 
\usepackage[margin=0.5in]{geometry}
\usepackage{amsmath}
\usepackage{amsfonts} 
\usepackage{enumerate} 

\DeclareMathOperator{\Ker}{Ker}
\DeclareMathOperator{\Ima}{Im}
\DeclareMathOperator{\Hom}{Hom}
\DeclareMathOperator{\GL}{GL}
\DeclareMathOperator{\sgn}{sgn}
\DeclareMathOperator{\rang}{rang}
\DeclareMathOperator{\q}{q}

\title{Linearna Algebra Cheat Sheet}
\author{Andrija Urosevic}

\begin{document}

\maketitle

\begin{multicols}{2}

\section{Uvod}

\subsection{Skupovi. Relacije. Funkcije}

    Za dati \textit{element} $a$ skupa $A$, pisemo $a \in A$, tj.\ element 
    $a$ pripada skupu $A$.
    Skup $B$ je \textit{podskup skupa} $A$, ako je svaki element skupa $B$ 
    ujedno i element skupa $A$, tj ako iz $x \in B$ sledi da je $x \in A$, 
    zapisuje se $B \subseteq A$.
    \textit{Dekartov proizvod} dva skupa $A$ i $B$ je skup
    \[A \times B = \{(a, b) | a \in A, b \in B\},\]
    pri cemu su \textit{uredjeni parovi} $(a_1, b_1)$ i $(a_2, b_2)$ jednaki 
    akko je $a_1 = a_2$ i $b_1 = b_2$

    Neka su $A$ i $B$ neprazni skupovi tada je \textit{funkcija} 
    (\textit{preslikavanje}) uredjenja troja $(A, B, f)$ cije prve dve 
    komponente su dati skupovi $A$ i $B$, a treca je `zakon' $f$ kojem se 
    svakom elmentu skupa $A$ dodeljuje tacno jedan element skupa $B$. Uredjena 
    trojka $(A, B, f)$ se zapisuje sa $f$ ili kao $f:A \mapsto B$.
    Skup $A$ zove se \textit{domen funkcije}, a $B$ njen \textit{kodomen}.

    Za funkciju $f : A \mapsto B$ kazemo da je \textit{injekcija} ili `1--1' 
    ako za bilo koja dva razlicita elementa $x_1, x_2 \in A$ i njihove slike 
    su razlicite tj.\ $f(x_1) \neq f(x_2)$. Formalnije:
    \[(\forall x_1, x_2 \in A) (x1 \neq x_2) \implies (f(x_1) \neq f(x_2))\]
    ili
    \[(\forall x_1, x_2 \in A) (x_1 = x_2) \implies (f(x_1) = f(x_2))\]

    Za funkciju $f : A \mapsto B$ kazemo da je \textit{surjekcija} ili `na' 
    ako za svaki $b \in B$ postoji $x \in A$ takav da je $b = f(a)$. 
    Formalnije:
    \[(\forall b \in B) (\exists a \in A)\text{ tako da } b = f(a)\]

    Funkcija koja je istovremeno bijekcija i surjekcija naziva se 
    \textit{bijekcija}.

    Neka je data funkcije $f : A \mapsto B$, i neka je $X \subseteq A$ tada 
    skup
    \[f(X) = \{y \in B | y = f(x), x \in X\} = 
    \{f(x) | x \in X\} \subseteq B\]
    nazivamo \textit{slikom skupa} $X$. Ako je $X = A$ tada za $f(A)$ 
    koristimo oznaku $\Ima A$ i nazivamo ga \textit{slikom funkcije} $f$.
    Funkcije $f$ je `na' akko je $\Ima A = B$. Analogno, za $Y \subseteq B$ 
    skup
    \[f^{-1}(Y) = \{y \in A | f(y) \in Y\} \subseteq A\]
    nazivamo preslikom skupa $Y$.

    \textit{Relacija} $\rho$ na skupu $A$ je svaki podskup Dekartovog proizvoda
    \[A^n = A \times A \times \cdots \times A.\]
    \textit{Binarna relacije} su podskupovi skupa $A \times A$. 
    Za $(a, b) \in \rho$ koristi se i oznaka $a \rho b$.

    Relacija $\rho$ je \textit{relacija ekvivalencije} 
    ako je refleksivna, simetricna i tranzitivna. 
    Relacija $\rho$ je \textit{relacija parcijalnog poretka} 
    ako je refleksivna, antisimetricna i tranzitivna.

\subsection{Grupe. Prsten. Polja}

    Svaki uređeni par $(S, *)$ skupa $S$ i bilo koje od binarnih operacija $*$ 
    u tom skupu zovemo i jednim \textit{grupoidom}. Ako je ta operacija 
    asocijativna, tj važi $(a * b) * c = a * (b * c)$ kažemo da je taj grupoid 
    \textit{asocijativan}. Asocijativne grupide zovemo i \textit{polugrupama}.

    Za element $e \in M$ kažemo da je \textit{neutral} polugrupe $(M, *)$ ili 
    same operacije $*$, ako za svako $a \in M$ važi $a * e = e * a = a$. 
    Polugrupe sa neutralom zovemo i \textit{monoid}.

    Element $a$ je \textit{invertibilan} ako postoji bar jedno $a^- \in M$ za 
    koje važi $a * a^- = e$ i $a^- * a = e$.

    Monoid u kojima su svi elementi inverzibilni zovemo i \textit{grupom}. 
    Ako je ona i komutativna nazivamo je \textit{abelovom grupom}.

    Za preslikavanje $f : G \mapsto K$ kažemo da je \textit{homomorfizam} 
    grupoida $(G, *)$ u grupoid $(K, \circ)$ ako za svako $a, b \in G$ važi 
    $f(a * b) = f(a) \circ f(b)$

    Strukturu $(K, +, \cdot)$ nazivamo \textit{prstenom} ako je:
    \begin{enumerate}
        \itemsep0em
        \item $(K, +)$ je komutativna/Abelova grupa 
        \item $(K, \cdot)$ je monoid
        \item Operacija $\cdot$ je distrbutivna na operaciju $+$
    \end{enumerate}

    Strukturu $F = (F, +, \cdot)$ nazivamo \textit{polje} ako je:
    \begin{enumerate}
        \itemsep0em
        \item $(F, +)$ je Abelova grupa
        \item $(F \backslash \{0\}, \cdot)$ je Abelova grupa
        \item Operacija $\cdot$ je distributivana na operaciju $+$
    \end{enumerate}

    Polje racionalnih brojeva $\mathbb{Q} = \{ \frac{n}{m} | n \in \mathbb{Z} 
    \text{ i } m \in \mathbb{N}\}$.

    Polje realnih brojeva $\mathbb{R}$.

    Polje kompleksnih brojeva $\mathbb{C} = \{ a + ib | a,b \in \mathbb{R} \}$.

\section{Vektorski Prostori}
    
    \textit{Vektorski prostor nad nad poljem $F$} je uređena trojka 
    $V = (V, +, \cdot)$ koja sadrži skup $V$ i dve binarne operacije 
    (sabiranje i množenje skalarom respektivno)
    \[+ : V \times V \mapsto V, (x, y) \mapsto x + y\]
    \[\cdot : F \times V \mapsto V, (a, x) \mapsto ax\]
    tako da za svako $u,v \in V$ i svako $a, b \in F$ važi:
    \begin{enumerate}
    \itemsep0em
    \item [V.1] $(V, +)$ je Abelova grupa
    \item [V.2] $a(u + v) = au + av$
    \item [V.3] $(a + b)u = au + bu$
    \item [V.4] $a(bu) = (ab)u$
    \item [V.5] $1u = u$
    \end{enumerate}

\subsection{Vektorski potprostor}
    
    Neka je $V$ vektorski prostor nad poljem $F$ i neka je $V' \subset V$. 
    Ako je $V'$ vektorski prostor zatvoren za sabiranje i množenje skalarom 
    nad $V$, onda kažemo da je $V'$ \textit{potprostor vektorskog prostora} $V$

    Ako su $U$ i $W$ potprostori vektorskog prostora V, onda je 
    $U + W$ \textit{suma} vektorskih potprostora $U$ i $W$:
    \[U + W = \{ u + w | u \in U \text{ i } w \in W \}\]

    Kažemo da je suma \textit{direktran} i obeležavamo je sa $U \oplus W$, 
    ako važi da je
    \[U + W = V \text{  i  } U \cap W = \{0\}\]

\subsection{Matrice}

    Neka je $F$ polje. Neka su $m$ i $n$ pozitivni celi brojevi. Onda je 
    pravougaoni niz elemenata iz $F$
    \[
        \begin{pmatrix}
            a_{11} & a_{12} & \cdots & a_{1n} \\
            a_{21} & a_{22} & \cdots & a_{2n} \\
            \vdots & \vdots & \ddots & \vdots \\
            a_{m1} & a_{m2} & \cdots & a_{mn}
        \end{pmatrix}
    \]
    naziva \textit{matrica} iz skupa $M_{mn}(F)$.


\subsection{Algebra}

    Neka je $V$ neki neprazan skup. Tada za strukturu 
    $V \cong (V, F, +, \cdot)$ kazemo da je F-\textit{algebra} ako je:
    \begin{enumerate}
        \itemsep0em
        \item [A.1] $(V, F, +, \cdot)$ vektorski prostor nad poljem $F$
        \item [A.2] $(V, +, *)$ prsten, tj.\ preslikavanje 
            $* : V \times V \mapsto V$ je \textit{mnozenje vektora}, 
            zadoboljava aksiome:
            \begin{enumerate} [(i)]
                \item $A * (B + C) = A * B + A * C$
                \item $(A + B) * C = A * C + B * C$
            \end{enumerate}
        \item [A.3] Preslikavanje $*$ je kompitabilno sa mnozenjem
            \[\lambda \cdot (A * B) = 
            (\lambda \cdot A) * B = 
            A * (\lambda \cdot B)\]
    \end{enumerate}


\subsection{Opšta Linearna Grupa}

    Grupa svih inverzibilnih elemenata endomorfizma vektorskog prostora $V$ 
    nad poljem $F$ zove se \textit{opsta linearna grupa} vektorskog 
    prostora $V$ nad poljem $F$ i opise se $\GL_F(V)$. 
    Dodatno, $\GL_F(V) \cong \GL_n(F)$

    Neka je matrica $A \in \mathbb{M}_n (F)$, onda je $A$ \textit{regularna} 
    ako je inverzibilna, tj.\ $A \in \GL_n(F)$, ako nije inverzibilna onda 
    je \textit{singularna}.

    Skup matrica koje se razlikuju od jediničnih matrica samo na 
    jednom elementu nazivaju se \textit{elementarne matrice}.

\subsection{Linearna kombinacija}

    Neka je $V$ vektorski prostor nad poljem $F$ i neka su 
    $v_1, v_2, \ldots ,v_n$ vektori iz $V$. Onda kažemo da je $v \in V$ 
    \textit{linearna kombinacija vektora} $v_1, v_2, \ldots ,v_n$ ako postoje 
    $a_1, a_2, \ldots , a_n \in F$ takvi da:
    \[v = a_1 v_1 + a_2 v_2 + \cdots + a_n v_n\]

    Neka je $S$ podskup vektorskog prostora $V$. Skup svih linearnih 
    kombinacija vektora $v_1, v_2, \ldots ,v_n \in S$ naziva se 
    \textit{linearni omotač} vektorskog prostora $V$, tj. 
    $S$ \textit{razapinje} $V$, formalno:
    \[V = \mathcal{L} (S) = \{ \sum_{i=1}^n a_i v_i | 
    a_i \in F, v_i \in S, i = 1, 2 \ldots n \}\]

\subsection{Baza i Dimenzija}

    Neka je dat skup $S = \{ v_1, v_2, \ldots, v_n \}$ koji je podskup 
    vektorskog prostora $V$. Kažemo da su vektori iz $S$ 
    \textit{linearno nezavisni} ako jednačina:
    \[a_1 v_1 + a_2 v_2 + \cdots + a_n v_n = 0, 
    \\ a_i \in F, i = 1, 2, \ldots, n\]
    ima trivijalno rešenje, tj.\ da je $a_1 = a_2 = \cdots = a_n = 0$. 
    Vektori su \textit{linearno zavisni} ako nisu linearno nezavisni.

    Konačan skup vektora, $\{e_1, \ldots, e_n\}$ je \textit{baza} vektorskog 
    prostora $V$ ako je $\mathcal{L} (\{e_1, \ldots, e_n\}) = V$ i ako su 
    $e_1, \dots, e_n$ linearno nezavisni.

    Neka je $V$ vektorski prostor nad poljem $F$. I neka su $e_1, \ldots, e_n$ 
    baze tog vektorskog prostora, kažemo da je \textit{dimenzija} vektorskog 
    prostora kardinalni broj neke njegove baze, tj.\ $\dim U = n$.

    Neka su $U$ i $W$ neki potprostori vektorskog prostora $V$ dimenzije $n$. 
    Tada važi:
    \[\dim U + \dim W = \dim(U + W) + \dim(U \cap W)\]

\subsection{Rang}

    Neka su $v_1, v_2, \ldots, v_m$ konačnodimenzionalni sistem vektora nekog 
    vektorskog prostora $V$. Kažemo da je \textit{rang} sistema vektora 
    $r$, tj.\ $ rang (v_1, v_2, \ldots, v_m) = r$ ako važi:
    \begin{enumerate}
        \itemsep0em
        \item Postoji linearno nezavisan podskup od 
            $\{v_1, v_2, \ldots, v_m \}$ koji sadrži tacno $r$ vektora
        \item Svaki podskup od $\{v_1, v_2, \ldots, v_m \}$, koji se sadrži 
            od $r + 1$ vektor je linearno zavisan
    \end{enumerate}

\section{Linearno Preslikavanje}

    Neka su $U$ i $W$ vektorski prostori nad poljem $F$. Onda je preslikavanje 
    $L: U \mapsto W$ \textit{linearno}, ako za svako $u, v \in U$ i svako 
    $\lambda \in F$ važi:
    \[ L(u + v) = L(u) + L(v) \text{  i  } L(\lambda u) = \lambda L(u)\]

    Linearno preslikavanje $L : U \mapsto W$, 
    koje je injektivno zovemo \textit{monomorfizam}, 
    sujektivno preslikavanje zovemo \textit{epimorfizam}, 
    i bijektivno preslikavanje \textit{homomorfizam}. 
    Ako je $U = W$ onda kažemo da je linearno preslikavanje 
    \textit{endomorfno}, bijekcije ednomorfizma su \textit{automorfizmi}.

\subsection{Izomorfizmi Vektorskih Prostora}

    Neka su $U$ i $W$ dva vektorska prostora nad poljem $F$. Kažemo da je $U$ 
    \textit{izomorfno} na $W$ ako postoji izomorfizam $f : U \mapsto W$. 
    Ako su $U$ i $W$ izomorfni pišemo $U \cong W$

\subsection{Regularni operatori}

    Linearni operator $L$ je \textit{regularan} ako je maksimalnog ranga.
    \begin{enumerate}[(i)]
        \itemsep0em
        \item ako je $\dim U \leq \dim V$ tada je $L$ \textit{regularan} 
            akko je $\dim \Ima L = \dim U$ i $\Ker L = \{0\}$.
        \item ako je $\dim U > \dim V$ tada je $L$ \textit{regularan} 
            akko je $\Ima A = V$.
    \end{enumerate}
    Za linearni operator $L$ kazemo da je \textit{singularan} ako je 
    $\Ker L \neq \{0\}$, tj.\ $d(L) \geq 1$.


\subsection{Teorema o Rangu i Defektu}

    Neka su $U$ i $W$ dva vektorska prostora nad $F$ i neka je 
    $L : U \mapsto W$. Onda je \textit{jezgro} od $L$:\ 
    $\Ker L= \{v \in U | L(v) = 0 \}$, \textit{slika} od $L$:\ 
    $\Ima L = \{ L(v) | v \in U\}$, \textit{rang} linearnog preslikavanja je 
    $\rho (L) = \dim \Ima L$, i \textit{defekt} linearnog preslikavanja 
    $\delta(L) = \dim \Ker L$.

    Neka su $U$ i $W$ dva vektorska prostora nad nekim poljem $F$. 
    Pretpostavimo da je $U$ konačno-dimenzionalan i neka je 
    $L : U \mapsto W$ linearno preslikavanja. Onda je $\Ker L$ i $\Ima L$ 
    konačnodimenzionalni potprostori od $U$ i $W$, i imamo jednakost:
    \[\rho (L) + \delta (L) = \dim U \]

\subsection{Vektorski prostor $\Hom_F (U, W)$}

    Neka su $U$ i $W$ dva vektorska prostora nad poljem $F$. Uređena četvorka 
    $\Hom(U, W) = (\Hom_F(U, W), F, +, \cdot)$, gde je
    \[ (A + B) (x) = A(x) + B(x), \text{ za } A, B \in Hom(U, V), 
    \text { i } \forall x \in U\]
    \[ (\lambda A) (x) = \lambda A(x), \text{ za } A \in Hom(U, V), 
    \lambda \in F, \text { i } \forall x \in U\]
    vektorski porprostor nad poljem $F$ svih linearnih preslikavanja 
    iz $U$ u $W$.

\subsection{Dualni prostor i preslikavanje}

   Neka je $V$ vektorski prostor nad poljem $F$. Neka je $F$ jednodimenzioni 
   vektorski prostor nad samim sobom. Skup svih linearnih preslikavanja 
   $V \mapsto F$ naziva se \textit{dualni prostor} i obelezava se sa $V^*$. Po 
   definiciji je:
   \[V^* = \mathcal{L} (V, F)\]


\section{Determinante}

\subsection{Grupa permutacija}

    \textit{Permutacija} skupa $S_n = \{1, 2, \ldots, n\}$ je bijektivno 
    preslikavanje
    \[\sigma : \{1, 2, \ldots, n\} \mapsto \{1, 2, \ldots, n\}.\]
    $(S_, \circ)$ je grupa koja se naziva \textit{grupa permutacija} ili
    \textit{simetricna grupa}. Grupa permutacija ima $\| S_n \| = n$ 
    elemenata.
    \[
        \sigma = 
        \begin{pmatrix}
            1 & 2 & \cdots & n \\
            \sigma(1) & \sigma(2) & \cdots & \sigma(n)
        \end{pmatrix}
    \]

\subsection{Parnost permutacija}

    Neka je $\sigma$ permutacije. Ako je $i < j$ i $\sigma(i) > \sigma(j)$, 
    uredjeni par $(i, j)$ ili uredjeni par $(\sigma(i), \sigma(j))$ se naziva 
    \textit{inverzija permutacije} $\sigma$. Broj inverzija se obelezava sa 
    $I(\sigma)$.

    Permutacija je parna ako je broj inverzija $I(\sigma)$ paran broj, a 
    neparna ako je $I(\sigma)$ neparan broj.

    \textit{Znak} permutacije $\sigma$ je definisan sa $+1$ ako je permutacija 
    parna i sa $-1$ ako je permutacija neprana. Obelezava se sa $\sgn(\sigma)$
    i moze da se eksplicitno izrazi kao:
    \[\sgn({\sigma}) = {(-1)}^{I(\sigma)}.\]

\subsection{Ciklus}

    Neka je $\sigma$ neka permutacija iz $S_n$. Permutacija $\sigma$ je 
    \textit{ciklus duzina} $k$ ako postoji uredjena $k$-torka:
    $(i_1, i_2, \ldots, i_k)$, gde su $i_1, i_2, \ldots, i_k \in S_n$ 
    takvi da je:
    \[\sigma(i_1) = i_2, \sigma(i_2) = i_3, \ldots, \sigma(i_k) = i_l\]
    \[\sigma(j) = j, \forall j \in S_n \backslash \{i_1, i_2, \ldots, i_k\}\]
    
    Ciklus duzine 2 zove se \textit{transpozicija}.

\subsection{Definicija determinante}

    Neka je dato jedinstveno linearno preslikavanje 
    $\det : \mathbb{M}_n (F) \mapsto F$, ako je $A \in \mathbb{M}_n(F)$ onda 
    $\det(A)$ nazivamo \textit{determinanta} matrice $A$. Determinante se 
    može zapisati kao:

    \[
        \det(A) = 
        \begin{vmatrix}
            a_{11} & \cdots & a_{1n} \\
            \vdots & &        \vdots \\
            a_{n1} & \cdots & a_{nn} 
        \end{vmatrix}
    \]
    \[ 
        \det(A) = \sum_{\sigma \in S_n} 
        \left( 
        \sgn(\sigma) \prod_{i=0} a_{i,\sigma(i)} 
        \right) 
    \]

\subsection{Multilinearnost determinante}

    Neka su $V_1, \ldots V_n$ i $W$ vektorski prostori nad poljem $F$. 
    Za preslikavanje 
    \[\varphi : V_1 \times V_2 \times \cdots \times V_n \mapsto W\]
    kazemo da je \textit{multilinearno} ako $\forall i=1,\ldots,n, 
    \lambda \in F$ i $a_i, a_i' \in V_i$,
    vaze sledeci uslovi:
    \begin{align*}
        \varphi [ a_1, \ldots, \lambda a_i + a_i', \ldots, a_n ] = 
        \lambda & \varphi [ a_1, \ldots, a_i, \ldots, a_n ] + \\
        &\varphi [ a_1, \ldots, a_i', \ldots, a_n ]
    \end{align*}

    Multilinearno preslikavanje za koje vazi da menja znak, ako 
    zamenimo bilo koje dve vrste (kolone) zove se 
    \textit{alternirajuce preslikavanje}, tj.\ za svako 
    $i < j$, $i, j = 1, 2, \ldots, n$, vazi:
    \[
        \varphi [ a_1, \ldots, a_i, \ldots, a_j, \ldots, a_n ] =
        - \varphi [ a_1, \ldots, a_j, \ldots, a_i, \ldots, a_n ]
    \]

    Determinanta je \textit{multilinearni alternirajuci funkcional}.


\section{Sistem Linearnih Jednacina}

    Neka je dat skup $F = \{f_1, \ldots, f_n\} \subset F[ x_1, \ldots, x_m ]$
    linearnih polinoma u promenljivima $x_1, \ldots x_m$. Skup $F$ definise 
    sledeci \textit{sistem linearnih jednacina}:
    \[f_1(x_1, \ldots, x_m) = a_{11}x_1 + \cdots + a_{1m}x_m - b_1 = 0\]
    \[\vdots\]
    \[f_n(x_n, \ldots, x_m) = a_{n1}x_1 + \cdots + a_{nm}x_m - b_n = 0,\]
    ili krace kao:
    \[\sum_{k=1}^m a_{ik}x_k= b_i, i=1,\ldots,n.\]
    Sistem se moze zapisati u obliku matrica kao $Ax = b$, gde je
    \[
        A = 
        \begin{bmatrix}
            a_{11} & \cdots & a_{1m} \\
            \vdots & &        \vdots \\
            a_{n1} & \cdots & a_{nm} 
        \end{bmatrix},
        x = 
        \begin{bmatrix}
            x_1     \\
            \vdots  \\
            x_m
        \end{bmatrix},
        b = 
        \begin{bmatrix}
            b_1     \\
            \vdots  \\
            b_n
        \end{bmatrix}
    \]

\section{Redukcija linearnog operatora na konacnodimenzionim prostorima}

\subsection{Prsten polinoma}

    \textit{Polinom} nad poljem $F$ je izraz:
    \[
        f(x) = a_n x^n + a_{n-1} x^{n-1} + \cdots + a_1 x + a_0, 
        a_i \in F, a_n \neq 0
    \]
    Skalari $a_n, \ldots, a_0$ polinomna nazivaju se \textit{koeficijenti} 
    polinoma $f$.

\subsection{Invarijantni potprostori}

    Neka je $A \in \Hom V$ linearni operator i neka je $L \subseteq V$ 
    prostora od $V$. Kazemo da je $L$ \textit{invarijantan prosotr} operatora 
    $A$ ako je $A(L) \subseteq L$.

\subsection{Karakteristicki polinom i invarijante slicnosti}

    Neka je $A \in M_n(F)$ proizvoljna matrica, tada matricu 
    $C = A - \lambda I_n$, gde je $\lambda$ promenljiva, naziva 
    \textit{karakteristicna matrica} od $A$. A njena determinanta
    \[
        \det (A - \lambda I_n) = 
        \begin{vmatrix}
            a_{11} - \lambda & a_{12} & \cdots & a_{1n} \\
            a_{21} & a_{22} - \lambda & \cdots & a_{2n} \\
            \vdots & \vdots & \ddots & \vdots \\
            a_{n1} & a_{n2} & \cdots & a_{nn} - \lambda
        \end{vmatrix}
        = p_A (\lambda)
    \]
    nazivamo \textit{karakteristicni polinom} matrice $A$.

    Karakteristicni polinom je inverijanta slicnosti.

\subsection{Hamilton---Kejlijeva teorema}

    Neka je $A \in M_n(F)$ proizvoljna matrica, tada je $p_A (A) = 0$, tj.\ 
    svaka matrica ponistava svoj karakteristicni polinom.

\subsection{Minimalni polinom matrice}

    \textit{Minimalni polinom} matrice $A$ je monicni polinom najmanjeg 
    stepena kojeg ponistava matrica $A \in M_n(F)$, tj.\ vazi:
    \[\mu_A (A) = 0\]

\subsection{Sopstvene vrednosti linearnog operatora}

    Neka je $V$ vektorski prostor i neka je $A : V \mapsto V$ linearno 
    preslikavanje vektorskog prostora $V$ u samo sebe. Element $v \in V$ se 
    naziva \textit{sopstveni vektor} preslikavanja $A$, ako postoji neko 
    $\lambda$ tako da $Av = \lambda v$. Ako je $v \neq 0$ tada je $\lambda$ 
    \textit{jedinstveno odredjeno}. U tom slucaju $\lambda$ je 
    \textit{sopstvena vrednost} preslikavanja $A$.

\subsection{Nilpotentni operatori}

    Za operator $A \in \Hom V$, kazemo da je \textit{nilpotentan} ako postoji
    $r \in \mathbb{N}$ takva da je $A^r = 0$. Najmanji prirodni broj $k$ za 
    kojeg vazi da je $A^r = 0$, ali $A^{k-1} \neq 0$ zove se 
    \textit{indeks nilpotentnosti} operatora $A$.

\subsection{Zordanova normalna forma}


\section{Unitarni Prostori}

    Neka je $V$ kompleksni vektorski prostor, kazemo da je $V$ 
    \textit{unitaran} vektorski prostor ako je definisana funkcije 
    $\langle \cdot, \cdot \rangle : V \times V \mapsto \mathbb{C}$ 
    za koju vaze sledece aksiome 
    ($\forall v_1, v_2, w \in V, \forall \lambda \in \mathbb{C}$):
    \begin{enumerate}
        \itemsep0em
        \item [U.1] $\langle v_1 + v_2, w \rangle = 
            \langle v_1, w \rangle + \langle v_2, w \rangle$ 
            (aditivnost po 1.-om)
        \item [U.2] $\langle \lambda v, w \rangle = 
            \lambda \langle v, w \rangle$ (homogenost po 1.-om)
        \item [U.3] $\langle v, w \rangle = \lambda \langle w, v \rangle$ 
            (hermitska simetricnost)
        \item [U.4] $\langle v, v \rangle \geq 0$ (pozitivna definitnost)
        \item [U.5] $\langle v, v \rangle = 0$ akko $v = 0$ (nedegenerisanost)
    \end{enumerate}

    Funkcija koja zadovoljava aksiome (U.1--U.3) zove se 
    \textit{hermitski bilinearni funkcional}, ako zadovoljava aksimu 
    nedegenerisanost kazemo da je 
    \textit{nedegenerisana hermitska bilinearna forma}, a ako jos zadoboljava 
    i pozitivnu definitnost onda takvu funkciju zovemo 
    \textit{skalarni proizvod} na $V$.

\subsection{Skalarni proizvodi}

    Neka je $V$ vektorski prostor nad poljem $F$. \textit{Skalarni proizvod}  
    nad $V$ je dodela koja paru elementata $v, w \in V$ dodeljuje skalar.
    Obelezava se sa $\langle v, w \rangle$ ili sa $v \cdot w$ i zadovoljava 
    svojstva:
    \begin{enumerate}
        \itemsep0em
        \item [S.1] $\langle v, w \rangle$ = $\langle w, v \rangle$ za svaki 
            $v, w \in V$
        \item [S.2] $\langle u, v + w \rangle = 
            \langle u, v \rangle + \langle u, w \rangle$, za $u,v,w \in V$
        \item [S.3] $\langle \lambda u, v \rangle = 
            \lambda \langle u, v \rangle$ 
            i $\langle u, \lambda v \rangle = \lambda \langle u, v \rangle$, 
            za $\lambda \in F$
    \end{enumerate}

\subsection{Ortogonalnost}

    Neka je $V$ unitarni prostor, za vektore $v$ i $w$ kazemo da su 
    \textit{ortogonalni} (\textit{normalni}) i pisemo $v \perp w$ ako je 
    $\langle v, w \rangle = 0$.
    Za bilo koja dva potprosotra $W_1, W_2 \subseteq V$, kazemo da je $W_1$ 
    \textit{ortogonalan} na $W_2$ akko
    \[
        \langle W_1, W_2 \rangle = 0, \text{ tj.\ }\langle w_1, w_2 \rangle = 0
    \]
    za svako $w_1 \in W_1$, i za svako $w_2 \in W_2$ i pisemo $W_1 \perp W_2$.

\subsection{Norma. Ugao. Kosi---Svarcova nejednakost}

    U unitarnom prostoru funkcije $\| \cdot \| : V \mapsto R_o^+$, definisana
    $\|v\|=sqrt(\langle v, v \rangle)$ naziva se \textit{norma}.

    Neka su $v, w$ bilo koja dva vektora unitarnog prostora $V$. Tada vazi 
    nejednakost (Kosi---Svarcova nejednakost)
    \[ | \langle v, w \rangle | \leq \|v\| \|w\| \]
    pri cemu jednakost vazi akko su vektori $v$ i $w$ linearno zavisni.

    U unitarnom prostoru definisan je i pojam \textit{ugla} za bilo koja dva 
    ne-nula vektora $v$ i $w$ formulom:
    \[
        \cos \angle (v, w) = \frac{\langle v, w \rangle}{\|v\| \|w\|}
    \]

\subsection{Gram---Smitova ortagonalizacija}

    Neka je $V$ unitaran prostor dimenzije $n$, za bazu 
    $e = (e_1, e_2, \ldots, e_n)$ kazemo da je \textit{ortonormirana} ako vazi:
    \[\langle e_i, e_j \rangle = \delta_{ij}, \\ \forall i,j = 1,2,\ldots, n\]

    \textit{Gram---Smitov postupak ortogonalizacije} je algoritam kojim se 
    neka proizvoljna baza $f$ vektorskog prostora $V$ zamenjuje ortonormiranom 
    bazom $e$ i to tako da se u svakom koraku podudaraju linearni omotaci 
    $\mathcal{L}(\{ f_1, \ldots, f_i\}) = \mathcal{L}(\{ e_1, \ldots, e_i \})$

    \begin{enumerate}
        \itemsep0em
        \item $e_1 = \epsilon_1 \frac{f_1}{\| f_1 \|}$, gde je 
            $\epsilon_1 \in \{-1, 1\}$
        \item $v_2 = f_2 - \langle f_2, e_1 \rangle e_1$, pa onda
            $e_2 = \epsilon_2 \frac{v_2}{\| v_2 \|}$, gde je 
            $\epsilon_2 \in \{-1, 1\}$
        \item $v_k = f_k - \sum_{i=1}^k \langle f_k, e_i \rangle e_i$, pa onda
            $e_k = \epsilon_k \frac{v_k}{\| v_k \|}$, gde je 
            $\epsilon_k \in \{-1, 1\}$
    \end{enumerate}

    Neka je $f = (f_1, f_2, \ldots, f_n)$ baza unitarnog vektorskog prostora 
    $V$. Tada postoji pozitivno orijentisana ortonormirana baza 
    $e=(e_1, e_2, \ldots, e_n)$ takvda da je
    \begin{align*}
        & f_1 = a_{11} e_1 \\
        & f_2 = a_{12} e_1 + a_{22} e_2 \\
        & \vdots \\
        & f_n = a_{1n} e_1 + a_{2n} e_2 + \cdots + a_{nn} e_n
    \end{align*}

\section{Bilinearne i kvadratne forme}

    \textit{Bilinearni funkcional} na $F$ je preslikavanje 
    $A: V \times V \mapsto F$ ($F \in \{\mathbb{C}, \mathbb{R}\}$) za koje 
    vaze sledece aksiome, za sve $v_1, v_2, w \in V$ i $a, b \in F$:
    \begin{enumerate}
        \itemsep0em
        \item [B.1] $A (a v_1 + b v_2, w) = a A (v_1, w) + b A (v_2, w)$
        \item [B.2] $A (w, a v_1 + b v_2) = a A (w, v_1) + b A (w, v_2)$
    \end{enumerate}

    Bilinearni funkcional $A$ je \textit{hermitski} ako uz aksiomu [B.1] 
    vazi jos i aksioma:
    \begin{enumerate}
        \itemsep0em
        \item [B.3] $A (v, w) = \overline{A (w, v)}$
    \end{enumerate}

\subsection{Kongruentnost matrica. Rang i Jezgro hermitskog bilinearnog 
            funkcionala}

    Za dve matrice $A, B \in M_n(F)$ kazemo da su \textit{kongruentne}, tj.\ 
    $A \equiv B$ akko postoji regularna matrica $P \in M_n(F)$ takva da je
    \[B = P^{*} A P\]

    \textit{Rang hermitskog bilinearnog funkcionala} je broj ranga matrice 
    hermitskog bilinearnog funkcionala $A(e)$, u proizvoljnoj bazi $e$. Kazemo 
    da je funkcional regularan ako je $\rang (A(e)) = \dim V$.

    Neka je $A(v, w)$ bilinearni funkcional na vektorskom prosturu $V$. Skup 
    $N_A = \{v \in V | A(v, w) = 0, \forall w \in V\}$ nazivamo 
    \textit{jezgrom} bilinearnog funkcionala $A$.

\subsection{Kvadratne forme}

    Neka je $A(v, w)$ \textit{simetricni} bilinearni funkcional, onda je 
    formulom $\q (v) = A(v, v)$ definisano preslikavanje $\q : V \mapsto F$ 
    koje se zove \textit{kvadratna forma}. $A(v, w)$ zove se 
    \textit{polarna forma} kvadratne forme $\q$.

    Ako je $A(v, w)$ hermitska bilinearna forma onda pridruzeni kvadratni 
    funkcional nazivamo \textit{hermitski kvadratni funkcional}.

    Za hermitsku kvadratnu formu $\q (v) = A(v, v)$ kazemo da je 
    \textit{pozitina} ako je $\q (v) \geq 0$, $\forall v \in V$, dok je 
    \textit{strogo pozitivna forma} ako vazi $\q (v) = 0$ 
    (ili $(\forall v \neq 0)(\q (v) > 0)$).


\end{multicols}

\end{document}
