\documentclass[12p,a4paper]{article}

\usepackage[serbian]{babel} 
\usepackage[T2A]{fontenc} 
\usepackage[utf8]{inputenc} 
\usepackage{amsthm}
\usepackage{multicol} 
\usepackage[margin=0.5in]{geometry} 
\usepackage{amsmath} 
\usepackage{amsfonts} 
\usepackage{enumerate} 
\usepackage{amssymb}
\usepackage{tikz}
\usepackage{booktabs} 
\usepackage{graphicx} 
\usepackage{listings} 

\DeclareMathOperator{\Dom}{Dom} 
\DeclareMathOperator{\Ima}{Im} 
\DeclareMathOperator{\nzd}{NZD} 
\DeclareMathOperator{\nzs}{NZS} 
\DeclareMathOperator{\NOR}{NOR} 

\newtheorem*{theorem}{Teorema}
\newtheorem*{prop}{Tvrđenje}

\lstset{language=C}

\title{Uvod u Organizaciju i Arhitekturu Racunara 2B --- Cheat Sheet}
\author{Andrija Urošević}

\begin{document}

\maketitle

\begin{multicols}{2}


    \section{Fon-Nojmanovi Racunari}

    \subsection{ISA}

    \textbf{ISA} (eng. \emph{Instruction Set Architecture}) je jedan apstraktan
    model racunara. ISA obuhvata skup registara, skup instrukcija, formati 
    instrukcija, nacini adresiranja operanada, velicina memorijskog prostora,
    rezim rada, sistem prekida itd.

    \subsection{Instrukcioni Formati}

    Struktura masinske instrukcije:
    \begin{lstlisting}
        OPCODE | OPERANDS  
    \end{lstlisting}

    Operandi mogu biti registarski, memorijski i neposredni.
    Razmotrimo sledeci izraz u zavisnosti od vrste instrukcije: 
    $(a + b) * (c + d)$

    \subsubsection{Tro-adresne Instrukcije}

    \begin{lstlisting}
        OPCODE | DEST | SRC | SRC
    \end{lstlisting}
    \begin{lstlisting}
    ADD     R1, A, B
    ADD     R2, C, D
    MUL     X, R1, R2
    \end{lstlisting}

    Imaju tri adresna polja za specifikovanje registra ili memorijske lokacije.
    Program je mnogo manji po duzini, ali broj bitova po instrukciji se 
    povecava, sto ne znaci da ce biti brzi jer puno stvari mora da uradi
    u jednoj instrukciji (load, execute, store, itd.).

    \subsubsection{Dvo-adresne Instrukcije}

    \begin{lstlisting}
        OPCODE | DEST | SRC
    \end{lstlisting}
    \begin{lstlisting}
    MOV     R1, A
    ADD     R1, B
    MOV     R2, C
    ADD     R2, D
    MUL     R1, R2
    MOV     X, R1
    \end{lstlisting}

    Najpristupljeniji u komercijalnim racunarima.

    \subsubsection{Jedno-adresne Instrukcije}

    \begin{lstlisting}
        OPCODE | OPERAND
    \end{lstlisting}
    \begin{lstlisting}
    LOAD    A
    ADD     B
    STORE   T
    LOAD    C
    ADD     D
    MUL     T
    STORE   X
    \end{lstlisting}

    Koristi \emph{akumulator} registar za manimulisanjem podacima. Jedan
    operadn je akumulator a drugi je u registru ili na memorijskoj lokaciji.

    \subsubsection{Nula-adresne Instrukcije}

    \begin{lstlisting}
    PUSH    A
    PUSH    B
    ADD
    PUSH    C
    PUSH    D
    MUL
    POP     X
    \end{lstlisting}

    Ovi racunari su bazirani na staku. Da bi se izracunao izraz prvo se 
    prevede u obrnutu poljsku notaciju.
    
    \subsection{Load/Store Arhitektura}

    \emph{Load/Store arhitektura} je ISA koja deli instrukcije u dve 
    kategorije: pristupanje memoriji (load/store izmedju memorije i registra)
    i ALU operacije (koje se izvrsavaju samo nad registarima). Implementira
    ih RISC.\
    
    Na primer, oba operanda i destinacija za ADD operaciju moraju biti u 
    registrima.
    
    \subsection{RISC/CISC}

    \textbf{CISC} (eng. \emph{Complex Instruction Set Computer}):
    \begin{itemize}
        \itemsep0em
        \item Veliki broj instrukcija
        \item Komplekse instrukcije promenljive duzine
        \item Instrukcije mogu trajati vise otkucaja radnog takta
        \item ISA radi sto je vise moguce koriscenjem hardvera
        \item Efikasno koriscenje RAM-a
    \end{itemize}

    \textbf{RISC} (eng. \emph{Reduced Instruction Set Computer}):
    \begin{itemize}
        \itemsep0em
        \item Mali broj instrukcija
        \item Jednostavne instrukcije fiksirane duzine
        \item Instrukcije traju tacno jedan otkucaj radnog takta
        \item Kompajleri visokog nivoa preuzimaju kodiranje kompleksnih 
            instrukcija
        \item Tesko koriscenje RAM-a (izaziva problem uskog grla)
    \end{itemize}

    \subsection{Adresiranje}

    \subsubsection{Neposredno Adresiranje}

    Operand se zadaje najlakse ako se zada na mestu gde se zadaje adresa.
    Takav operand se zove \emph{neposredni operand}
    (eng. \emph{immediate operand}). 
    Primer:
    \begin{lstlisting}
        MOV     R1, 4
    \end{lstlisting}
    Koristi se u radu sa malim konstantama.

    \subsubsection{Direktno Adresiranje}

    Operand se nalazi na potpunoj adresi koja je zadata u instrukciji. 
    Koristi se u radu sa globalnim promenljivim, jer se adresa globalnih
    promenljiva ne menja dok njihova vrednost moze da se menja.

    \subsubsection{Registarko Adresiranje}

    Slicno je kao i direktno adresiranje sem sto se ovde ne zadaje 
    memorijska lokacija vec registar. Primer
    \begin{lstlisting}
        MOV     R1, R2
    \end{lstlisting}
    Prevodioci se trude da otkriju koje promenljive se najvise koriste i njih
    smestaju u registre.

    \subsubsection{Indirektno Adresiranje}

    Zadati operand dolazi iz memorije ili odlazi u nju, ali njegova adresa
    nije zabelezena unutar instrukcije kao u direktnom adresiranju, vec se 
    nalazi u registru.
    \begin{lstlisting}
        ADD     R1, [R2]    
    \end{lstlisting}

    \subsubsection{Indeksno Adresiranje}

    Adresiranje memorije tako sto se navede registar i konstantno rastojanje
    zove se \emph{indeksno adresiranje} (eng. \emph{indexed addressing})
    \begin{lstlisting}
        ADD     R1, [R2 + R3]
    \end{lstlisting}
    
    \subsubsection{Relativno Adresiranje}

    U instrukcijama grananja potrebna je i odredisna adresa koju takodje treba
    zadati. Vrednost operanda je relativna adresa u odnosu na adresu tekuce
    instrukcije, tj. PC registra.

    To je u stvari indeksno adresiranje, s registrom PC.\

    \section{x86--64 Arhitektura}

    \subsection{Nacini Adresiranje}

    \begin{lstlisting}
    MOV     rax, 10        
    MOV     rax, rbx        
    MOV     rax, 0x603829   
    MOV     rax, [rbx + rcx]
    MOV     rax, [rbx + 4 * rcx]
    MOV     rax, [rbx + 4 * rcx + 8]
    \end{lstlisting}

    \subsection{Tipovi Instrukcija}

    \subsubsection{Instrukcije Transfera}

    \begin{lstlisting}
    MOV     dst, src
    LEA     dst, src
    PUSH    src
    PULL    dst
    \end{lstlisting}

    \subsubsection{Aritmeticke i Bitovske Instrukcije}

    \begin{lstlisting}
    ADD     dst, src ;dst += src
    SUB     dst, src ;dst -= src
    IMUL    dst, src ;dst *= src
    NEG     dst      ;dst = -dst

    AND     dst, src ;dst &= src
    OR      dst, src ;dst |= src
    XOR     dst, src ;dst ^= src
    NOT     dst      ;dst =~dst

    SHL     dst, count ; dst <<= count
    SAR     dst, count ; dst >>= count
    SHR     dst, count ; dst >>= count

    MUL    src       ;rax *= src
    SHL     dst      ;dst <<= 1
    \end{lstlisting}

    \subsubsection{Flag-ovi}

    PSW registar je registar koji sadrzi informacije o stanju procesora.
    \begin{itemize}
        \itemsep0em
        \item \textbf{ZF} --- Nula (eng. \emph{Zero Flag}) pokazuje da li je
            rezultat aritmeticke ili logicke operacije bio nula.
        \item \textbf{CF} --- Prenos (eng. \emph{Carry Flag}) dopusta
            brojevima vecim od jedne redi da se dodaju ili oduzimaju, tako sto
            cuva prenos.
        \item \textbf{SF} --- Znak (eng. \emph{Sign Flag}) pokazuje da li je
            rezultat matematicke operacije negativan.
        \item \textbf{OF} --- Prekoracenje (eng. \emph{Overflow Flag}) pokazuje
            da li je oznaceni rezultat operacije prevelik da stane u registar.
    \end{itemize}
    
    \subsubsection{Instrukcije poredjenja}

    \begin{lstlisting}
    CMP     op1, op2 ;op1 - op2
    TEST    op1, op2 ;op1 & op2

    JMP     target  
    JE      target  ;(ZF=1)
    JNE     target  ;(ZF=0)
    JL      target  ;(SF!=OF)
    JLE     target  ;(ZF=1 ili SF!=OF)
    LG      target  ;(ZF=0 i SF=OF)
    LGE     target  ;(SF=OF)
    JA      target  ;(CF=0 i ZF=0)
    JB      target  ;(CF=1)
    JS      target  ;(SF=1)
    JNS     target  ;(SF=0)
    \end{lstlisting}
    
    \subsection{Funkcije}

    \subsubsection{Pozivanje Prodecura}
    
    Pre poziva procedure mora se sacuvati povratna adresa. Ona moze da se cuva
    u registar ili na stek.

    Ako bi postojao poseban registar koji bi cuvao povratnu adresu, nakon
    zavrsetka procedure znali bi smo gde da se vratimo. Ali recimo da 
    procedura P poziva proceduru Q i onda procedura Q poziva proceduru R, nakon
    izvrsavanja procedure R i vracanja u proceduru Q izgubili bi adresu za 
    vracanje u proceduru P.

    Druga mogucnost jeste cuvanje povratne adrese na steku. Pozivajuca 
    procedura gura adresu na stek, nakon izvrsavanja pozvane procedure ona
    skida adresu sa steka. Time se resava problem gubljenja povratne adrese
    i omogucava se rekurentan poziv procedure.
    
    \subsubsection{Prenos Argumenata}

    Argumenti se takodje mogu prenositi na dva nacina preko registara i preko
    steka.

    Argumenti se prenose preko registra na prethodno dogovoren nacin i to
    po sledecem redosledu: RDI, RSI, RDX, RCX, R8, R9. Ovaj metoda ne radi
    za pozivanje funkcije u trenutno pozvanoj funkciji i kada su parametri
    veci od 8 bajta.

    Argumenti mogu da se prenose i preko steka tako sto ih pozivajuca 
    procedura gurne na stek u obrnutom poretku, tj.\ prvo se gurne param n, 
    onda param (n-1), \ldots,
    param 1. Onda nakon izvrsavanja pozvane procedure, pozivajuca procedura
    skida sa steka prethodno dodate argumente.

    \subsubsection{Vracanje Vrednosti}

    Povratna vrednost ako je 8 bajta ili manje vraca se preko registra RAX,
    u slucaju da je vrednost sa pokretnim zarezom koristi se 
    \emph{floating-point register}.

    Vrednost ukoliko je neka strutura ona se gura na stek.
    
    \subsubsection{Pozivanje funkcije x86--64}

    Skeleton pozivanja funkcije:
    \begin{lstlisting}
function:
    # prologue
    PUSH    ebs         
    MOV     ebp, esp
    SUB     esp, N
    :::
    # epilogue
    MOV     ESP, EBP
    POP     EBP
    RET     0

main:
    :::
    MOV     RDI, 1st arg
    MOV     RSI, 2nd arg 
    :::
    MOV     R9, 6th arg
    PUSH    7th arg
    PUSH    8th arg
    :::
    CALL function
    POP     n-th arg
    :::
    POP     7th arg
    :::
    \end{lstlisting}
    
    \section{Procesor}
    
    Osnovne komponente procesora su putanja podataka, kontrolna jedinica, 
    memorijske komponente (registri i kes memorija), radni takt, logicke 
    kapije.

    \emph{Putanja podataka} (eng. \emph{data path}) je deo procesora koji 
    sadrzi aritmeticko-logicku jedinicu (ALI), njene ulaze i izlaze. Sadrzi
    i niz registara opste namene i memorijske registre

    \includegraphics[width=0.8\columnwidth]{Figures/datapath1.png}

    \includegraphics[width=0.8\columnwidth]{Figures/datapath2.png}

    \includegraphics[width=0.8\columnwidth]{Figures/datapath3.png}

    \subsection{Interni Registri Procesora}

    \begin{itemize}
        \itemsep0em
        \item PC --- programski brojac (eng. \emph{Program Counter})
        \item IR --- instrukcioni registra 
            (eng. \emph{Instruction Registar})   
        \item MAR --- registar memorijskih adresa 
            (eng. \emph{Memory Address Register})
        \item MDR --- registar memorijskih podataka
            (eng. \emph{Memory Data Register})
        \item PSW --- statusni registart (eng. \emph{Program Status Word})
        \item CW --- kontrolni registar (eng. \emph{Control Word})
    \end{itemize}

    \subsection{Faze Izvrsavanja Instrukcije}

    \begin{enumerate}
        \itemsep0em
        \item \emph{Dohvatanje instrukcije} (eng. \emph{Fetch}): Instrukcije 
            se ucitava iz memorije sa adrese na koju ukazuje PC registar i 
            prebacuje na MDR registar. U MAR registar se prebaci vrednost PC
            registra, izdaje se komanda za citanje memoriji. Vrednost dolazi
            sa magistrale podataka do MDR registra, i ona se u njemu zapamti.
        \item \emph{Dekodiranje instrukcije} (eng. \emph{Decode}): Instrukcija
            se prebacuje u IR registar cija se vrednost salje na ulaze 
            kontrolne jedinice. Dekodiranje se vrsti kada kontrolna jedinica
            utvrdjuje koju instrukciju zapravo izvrsava i koje korake treba
            da preduzme u nastavku.
        \item \emph{Izvrsavanje instrukcije} (eng. \emph{Execute}): Izvrsava
            se odredjenja operacija. 

            Ukoliko instrukcija sadrzi memorijske 
            operande, u ovoj fazi sve vrsi i dohvatanje operanada iz memorije
            (na slican nacin kao i dohvatanje instrukcije). 

            U slucaju aritmeticko-logickih instrukcija, vrednost operanada se
            propusta kroz ALU da bi se izracunao rezultat. 

            U slucaju transfera izmejdu registra, vrednost izvornog registre 
            se propusta korz ALU bez ikakve operacije i prebacuje u odredisni 
            registar.

            U slucaju transfera iz registra u memoriju, vrednost iz izvornog
            registra se kopira u MDR registra, a adresta odredista u MAR 
            registar. Nakon toga se preko magistrale podatak upisuje na 
            zadatu adresu.

            U Slucaju instrukcije skoka, izvrsavanje zavisi od ispunjenosti 
            uslova. Ukoliko flegovi ukazuju da je uslov ispunjen, vrsi se upis
            adrese na koju se skace u PC registar, u suprotnom ne radi se 
            nista.
    \end{enumerate}

    \subsection{Nacini Implementacije Kontrolne Jedinice}

    \subsubsection{Tvrdo Ozicena Implementacija}

    Tvrdo Ozicena Implementacija (eng. \emph{Hardwired Control Unit}) ili
    ``hardverska implementacija''.
    \begin{itemize}
        \itemsep0em
        \item Dizajnira se sekvencijalno kolo koje funkcionise kao 
            konacni trasduktor
        \item Na ulazu se nalaze vrednosti IR i PSW registra, a na izlazu
            se nalazi skup kontrolnih signala koji se salju putanji podataka,
            momoriji, i ostalim delovima racunara.
        \item Stanja odgovaraju fazama izvrsavanja instrukcija
        \item Mogu postojati i medjustanja u slucaju komplikovanih operacija.
        \item Pogodan za procesore sa jednostavnim skupom instrukcija (RISC).
        \item Brz je, ali zato ne moze lako da za modifikuje
    \end{itemize}

    \subsubsection{Mikroprogramirana Implementacija}

    Mikroprogramirana implementacija (eng. \emph{Microprogrammed Contrl Unit})
    ili ``softverska implementacija''.
    \begin{lstlisting}
        CONTROL SIGNAL | ADDRESS BITS
    \end{lstlisting}
    \begin{itemize}
        \itemsep0em
        \item Kontrolna jedinica sadrzi \emph{kontrolnu memoriju} 
            (eng. \emph{control store}). Svaka adresibilna lokacija u ovoj
            memoriji sadrzi jendu mikroinstrukciju.
        \item Svaka mikroinstrukcija se izvrsava u jednom ciklusu.
        \item Adresni bitovi iz tekuce mikroinstrukcije se kombinuju na 
            odgovarajuci nacin sa vrednoscu IR i PSW registra cime se dobija 
            adresa sledece mikroinstrukcije u kontrolnoj memoriji.
        \item MIR registar (eng. \emph{Micro Instruction Register}) sadrzi
            tekucu mikroinsrukciju.
        \item Niz mikroinstrukcija u kontrolnoj memoriji naziva se
            \emph{mikroprogram}. 
        \item PSW registar takodje utice na izbor mikroprograma koji ce se
            izvrsavati, jer se pojedinacne instrukcije izvrsavaju drugacije 
            u zavisnosti da li je niki uslov ispunjen ili ne.
        \item Mikroprogramirana kontrolna jedinica omoguca da se dizajn 
            kontrolne jedinice svede na mikroprogramiranje. Jednostavnost u
            slucaju slozenih skupova instrukcije (CISC).
        \item Losa strana je sto je sporiji od tvrdo-ozicenih implementacija
    \end{itemize}

    Struktura mikroinstrukcije:
    \begin{enumerate}
        \itemsep0em
        \item \emph{Horizontalna struktura}: Svi kontrolni signali postoje
            kao pojedinacni bitovi mikroinstrukcije i mogu se direktno
            usmeriti ka odgovarajucim komponentama procesora.
        \item \emph{Vertikalne strukture}: Ideja je da se duzina 
            mikroinstrukcije smanji tako sto se prepoznaju neke tipicne
            kombinacije kontrolnih signala koje se javljaju u 
            mikroinstrukcijama.
    \end{enumerate}

    \section{Memorija}

    \subsection{Karakteristike Memorije}

    \subsubsection{Vrste Pristupa}

    \paragraph{Proizvoljan Pristup:}
    Kod ovih memorija postoji mehanizam za pristupanje svakoj adresibilnoj
    jedinici u priblizno konstantnom vremenu. Koristi se jos i
    \emph{slucajan pristup} (eng. \emph{random access}). 

    \paragraph{Sekvencijalni Pristup:} 
    Adresibilnim jedinicama u momoriji se moze pristupiti iskljucivo redom. Da 
    bismo pristupili 10-toj adresibilnoj jedinici, moramo pre toga da 
    pristupimo svim adresibilnim jedinicamo od 0-te do 9-te redom.

    \paragraph{Direktan Pristup:} 
    Kod ovih memorija posotji mogucnost direktnog pristupa nekoj okolini
    adresibilne jedinice u memoriji na osnovu zadate adrese. Kada se pristupi
    okolini, pretraga trazene adresibilne jedinice se nastavlja sekvencijalno 
    unutar ukoline.

    \paragraph{Asocijativni Pristup:}
    Podatku se ne pristupa preko adrese, vec tako sto se zadaje neka maska
    koja se na neki nacin pretrazuje u memoriji.

    \subsubsection{Kapacitet}

    Kolicina podataka koja se moze sacuvati u memoriji. Izrazava se u
    (KiloByte, MegaByte, GigaByte, TeraByte, \ldots).

    \subsubsection{Brzina}
    
    Na brzinu uticu \emph{kasnjenje} (eng. \emph{memory latency}) i 
    \emph{brzina transfera} (eng. \emph{bandwitdth}). Kasnjene se 
    izrazava vremenom koje protekne od trenutka iniciranja transfera do
    trenutka kada taj transfer zaista zapocne. Brzina transfera se izrazava
    kolicinom podataka koji se mogu preneti u jednoj sekundi.

    \emph{Vreme pristupa} je prosecno ukupno vreme potrebno da se pojedinacni 
    podatak prenese iz memorije u procesor.

    \emph{Vreme memorijskog ciklusa} predstavlja vreme koje je potrebno da
    prodje izmedju dva uzastopna zahteva za pristupom. Ukljucuje vreme
    pristupa, ali i dodatno vreme koje mora proci nakon zavrsetka transfera
    do ponovnog zahteva za pristupom.
    
    \subsubsection{Postojanost}

    Memorija je \emph{postojana} (eng. \emph{persisten/non-valatile}) ako
    se iskljucenjem napajanja njen sadrzaj \underline{ne gubi}.

    Memorija je \emph{nepostojana} (eng. \emph{non-persistent/valatile}) ako
    se iskljucenjem napajanja njen sadrzaj \underline{gubi}.

    \subsubsection{Promenljivost}

    Memorija ciji je sadrzaj nepromenljiv nazivaju se \emph{nepromenljive 
    memorije} (eng. \emph{read-only memory}).

    Memorija ciji je sadrzaj promenljiv naziva se \emph{promenljiva memorija} 
    (eng. \emph{read-write memory}).

    \subsection{Memorijske Hijerarhije}

    \emph{Univerzalna memorija} je memorija koja ima sve pozeljne
    karakteristike. Ovakva memorija ne postoji.

    Memorije su organizovane u \emph{memorijsku hijerarhiju}.

    \includegraphics[width=\columnwidth]{Figures/memory_hierarchy.png}
    
    \subsubsection{Unutrasnje memorije}
    
    \paragraph{Read Only Memory (ROM):}
    ROM sadrzi programe i podatke koji su nepromenljivi i koji treba da su
    stalno prisutni da bi racunar mogao da radi. ROM ima proizvoljan pristup.
    Implementiraju se kao kombinatorno kolo.
    
    PROM (Programmable ROM) koji se moze programirati samo jednom, spaljivanjem
    odgovarajucih zica.

    EPROM (Erasable Programmable ROM) koji se moze programirati uz pomoc
    posebnog uredjaja za programiranje. Moze se programirati vise puta,
    ali je pre svakog programiranja potrebno obrisati prethodni sadrzaj.

    EEPROM (Electronically Erasable Programmable ROM) koji se moze brisati
    elektronskim putem i zatim ponovo programirati.

    \paragraph{Random Access Memory (RAM):}
    RAM memorija je osnovna memorija u kojoj se nalaze podaci i programi
    koji se trenutno izvrsavaju.

    SRAM (Staticki RAM) se konstruise pomocu kola slicnih osnovnim D 
    flip-flopovima. Cuva svoj sadrzaj sve dok postoji napajanje. 
    Vreme pristupanja nekoliko milisekundi.

    DRAM (Dinamicki RAM) se sastoji od niza celija, a svaka sadrzi jedan
    tranzistor i mali kondenzator. Kondenzatori se pune i prazne cime se 
    omogucava cuvanje logickih nula i jedinica. Kako elektricni napon
    tezi disipaciji, svaki bit u DRAM mora se \emph{osvezavati}. 

    SDRAM (Synchronous DRAM) je hibrid staticke i dinamicke memorije kojom
    upravlja glavni radni takt. Prednost je sto radni takt uklanja potrebu za
    upravljacke signale, sto je cini brzom.
    
    DDR (Double Data Rate) SDRM je memorija koja isporucuje podatke i na
    ulaznoj i na silaznoj ivici radnog takta.
    
    Da bi se smanjilo kasnjenje prilikom pristupa susednim memorijskim adresama
    je koriscenje \emph{isprepletanih memorija} 
    (eng. \emph{interleaved memory})
    Ideja je da se memorija podeli na nekoliko manjih memorijskih jedinica 
    koje nazivamo \emph{bankama}. Banke se organizuju slicno kao organizovanje
    memorijskih cipova u vece cipove. Nizi bitovi se koriste za izbor banke.
    Ovo daje vremena bankama da pripreme svoje podatke, kasnjenja pojadinacnih
    banki se medjusobno preklapaj. Ovaj metod radi samo ukoliko se pristupa 
    podacima sekvencijalno.

    \subsubsection{Spoljasnje memorije}

    \paragraph{Hard Disk:} Sadrze veci broj diskova premazanih magnetnim 
    materijalom koji su postavljeni na istu osovinu. Imaju veliki kapacitet,
    ali su dosta sporiji od najsporije unutrasnje memorije. Podeljen je na
    sektore cija je velicina 512 bajtova. Sektori se grupisu u staze koje
    predstavljaju koncentricne krugove na diskovima razlicitih precnika.
    Svaki sektor ima svoju adresu.

    \paragraph{Floppy Disk:} Slicni kao hard diskovi koji su elektromagnetni.
    Veoma malog kapaciteta. Danas se ne koriste.

    \paragraph{CD i DVD ROM:} Uredaju koji omogucuju samo citanje. Zapisane
    na optickom principu.

    \paragraph{Flash Memorije i SSD:} Zasnovani su na slicnoj tehnologiji
    kao i EEPROM, s razlikom sto ne omogucuju brisanje pojedinacnih bajtova,
    vec iskljucivo prisanje i upis citavih blokova. Dosta su brzi od
    hard diskova, ali su dosta skuplji a i zivotni ciklus im je dosta kraci.

    \subsection{Mapiranje Memorijskih Adresa}

    \emph{Adresni prostor} procesora je skup adresa koje procesor moze 
    direktno da adresira. Procesori se dizajniraju tako je adresni procesor
    veci nego sto je to zapravo potrebno (jer se pretpostavlja da ce se
    RAM uvecavati). Memorija ce zauzimati samo deo adresnog prostora procesora.
    \emph{Mapiranje memorijskih adresa} je pridruzivanje adrese procesorskog 
    adresnog procesora, adresibilnim lokacijama u memoriji.

    Mapiranje adresa se vrsi tako sto se adresa podeli na nizi i visi deo.
    Visi deo adrese aktivira memorijski modul u kome ce se traziti 
    odgovarajuca adresa, a nizi deo adrese odredjuje adresu u 
    okviru modula. 

    Jednostavan nacin za selektovanje memorijskog modula na osnovu viseg dela 
    adrese je da se odgovarajucom kombinatornom logikom implementira funkcija 
    koja ima vredno 1 samo za odgovarajucu kombinaciju visih bitova adrese.
    
    Ukoliko svakom memorijskom modulu odgovara jedinstvena kombinacija visih
    bitova adrese, tada ovakvo mapiranje nazivamo \emph{potpuno mapiranje}.

    Postoji mogucnost da se memorijski modul aktivira za veci broj kombinacija
    visih adresnih bitova. U tom slucaju je istom memorijskom modulu pridruzeno
    vise opsega adresa u okviru adresnog prostora procesora, te se svakoj 
    njegovoj memorijskoj lokaciji moze pristupiti pomocu razlicitih adresa.
    Ovaj nacin mapiranja zovemo \emph{parcijalno mapiranje} 

    \subsection{Memorijsko Poravnanje}
    
    Memorije su \emph{bajt-adresibilne} ako svaki bajt ima svoju adresku preko 
    koje mu se moze pristupiti. Medjutim, magistrala podataka je po pravilu 
    sira. Ono znaci da prilikom transfera prebacujemo vise od jednog bajta
    pocev od nekog na odredjenoj adresi. Zbog toga cesto postoji zahtev da se
    podaci u memoriji organizuju u blokove, tako da je pocetni bajt
    na adresi koja je deljiva nekim stepenom dvojke. Neke arhitekture to 
    zahtevaju, dok ce druge u slucaju da transfer nije adekvatno poravnat
    podeliti ga u 2 dela sto smanjuje performanse.

    \section{Kes Memorija}

    \emph{Kes memorija} se koristi kako bi se prevazisla razlika u brzini
    procesora i RAM-a. Ideja je da se podaci koji se najcesce koriste
    drze u kes memoriji koja je zasnovana na brzoj statickoj memoriji i time
    smanjilo kasnjenje.

    \paragraph{Princip Vremenske Lokalnosti:} (eng. \emph{temporal locality}) 
    Odnosi se na ponovno pristupanje nedavno posecenim memorijskim lokacijama.
    Memorijske lokacije blizu vrha steka ili instrukcije koje se ponavljaju u
    petlji.

    Koristi se uglavno tako sto se bira koji ce se sadrzaj iz kesa izbaciti
    tako sto se analizira koriscenje tog sadrzaja u bliskoj proslosti.
    
    \paragraph{Princim Prostorne Lokalnosti:} (eng. \emph{spatial locality})
    Odnosi se na adrese koje su veoma bliske adresama kojima je nedavno
    pristupano, pa je vrlo verovatno da ce se i njima pristupiti.

    Kes koristi ovo svojstvo da pripremi vise podataka nego sto se trenutno
    traziz, u nadi da ce oni uskoro biti iskorisceni.

    Kes memorija je podeljena na blokove 
    fiksirane velicine koje zovemo \emph{linije kesa}. Prilikom transfera
    podataka iz memorije u kes, uvek se ucitava ceo blok koji se upisuje u
    neku liniju kesa.

    Ukoliko se podatak, koji procesor zahteva, nalazi u nekoj od linija kesa 
    on se dohvata i transferuje u procesor, to se naziva \emph{pogotak kesa}
    (eng. \emph{cache hit}). 

    Ukolio se podatak ne nalazi u nekoj od linija kesa, to se naziva
    \emph{promasaj kesa} (eng. \emph{cache miss}).
    
    \subsection{Razdvojeni i Jedinstveni Kes} 

    Da li instrukcije i podatke treba drzati u istom kesu ili odvojeno?

    \emph{Jedinstveni kes} (eng. \emph{unified cache}), u kome se nalaze i
    instrukcije i podaci, jednostavan je i dobar je za velike memorije.

    \emph{Razdvojeni kes} (eng. \emph{split cache}), koji je podeljen
    na deo sa instrukcijama i deo sa podacima, je dobar za male memorije.

    \subsection{Nivoi Kesa}
    
    Kes memorija se obrazuje hjerarhijski, u vise nivoa. Svaki sledeci nivo je
    sve veci, ali je i sve sporiji. Na taj nacin se najcesce korisceni podaci 
    nalaze u najblizem kesu, oni malo redje u sledecem itd.
    
    Kod \emph{inkluzinvne organizacije} kesa svaki sledeci nivo kesa sadrzi 
    sve podatke koje je sadrzai i prethodni, ali i dodatne podatke koji u 
    prethodnom nisu postojali.

    Kod \emph{ekskluzivne organizacije} kesa sledeci nivoi sadrze iskljucivo
    druge podatke.

    \emph{L1 kes}: Nalazi se u cipu procesora. 
    Realizuje se kao razdvojeni 8-asocijativni kes.
    Tipicna velicina 32K + 32K.

    \emph{L2 kes}: Nalazi se u cipu procesora.
    Realizuje se kao jedinstveni 4-asocijativni kes.
    Tipicna velicina 256K.

    \emph{L3 kes}: Nalazi se na procesorskoj ploci i zajednicki je za
    sva jezgra procesora.
    Realizuje se kao jedinstveni 8/16-asocijativni kes.
    Tipicna velicina: 2M-4M.

    \subsection{Direktno Mapiranje}

    \begin{lstlisting}
ADRESA: | TAG | BROJ_LINIJE | POMERAJ |
    \end{lstlisting}

    POMERAJ:\ Predstavlja redni broj bajta unutar linije kesa.

    BROJ\_LINIJE:\ Predstavlja redni broj linije kesa u kes memoriji.

    TAG:\ Ostatak adrese

    \begin{lstlisting}
LINIJA KESA:
    | dirty_bit | valid_bit | TAG | PODACI |
    \end{lstlisting}


    \subsection{Asocijativno Mapiranje}

    \subsection{Set---Asocijativno Mapiranje}
    
    \subsection{Politika Zamene}

    Politika zamene definise nacin na koji se bira linija kesa koja ce 
    biti zamenjena i veoma je vazna zbog dobrog iskoriscenja vremenske
    lokalnosti. Kod direktnog mapiranje nema potebe za politiku zamene.
    U Slucaju asocijativnog mapiranja politika zamene treba da odabere jednu
    kes liniju iz skupa svih kes linija u kesu. U slucaju set---asocijativnog
    mapiranja, potrebno je odabrati jednu kes liniju iz odgovarajuceg skupa
    kes linija u koje dati podatak ide.

    \paragraph{LRU (Least Recently Used):}
    Zamenjuje se ona linija kesa koja najduze nije bila koriscena. Najbolji
    rezultati, najteza implementacija. Za $n$ kes linija moze se 
    implementirati konacni automat sa $n! $ stanja. Problem je sto $n! $ brzo
    raste.

    \paragraph{Pseudo---LRU:}
    Aproksimacija LRU politike koja se manje precizna, ali se lakse 
    implementira. Neka imamo n linija kesa. Ovaj skup delimo na dve polovine.
    Jednim bitom odredjujemo u kojoj se od te dve polovine nalazi linija kesa
    kojoj je najskorije pristupano. Na slican nacin se deli svaka od ovih
    polovina.

    \paragraph{FIFO (First In First Out):}
    One linije kesa koje su u kesu najduze najmanje se koriste, dok se one 
    koje su u kesu stigle kasnije vise koriste. Jednostavna je implementacija,
    dovoljno je cuvati brojac koji predstavlja odredjenu liniju kesa, i
    koji se svaki put uvecava kada se doda nova linija kesa.

    \paragraph{LFU (Least Frequently Used):}
    Najredje koriscena linije se izbacuje iz kesa. Potrebno je da postoji
    brojac za svaku liniju kesa, koji se uvecava pri svakom pristupu te 
    linije kesa, kao i slozena logika koja je potrebna za uporedjivanje
    brojaca. Implementacija je dosta slozena.

    \subsection{Politika Pisanja}

    \begin{enumerate}
        \itemsep0em
        \item \emph{Write-Back}: Ako se podatak nalazi u kesuo
        \item \emph{Write-Throught}:  
    \end{enumerate}

    \subsection{Odluke Pri Projektovanju Kesa}

    \paragraph{Velicina kesa:}
    Povecanjem kesa do neke granice imamo poboljsanje performansi, iznad
    te granice performanse opadaju.

    \paragraph{Velicina linije kesa:}
    Povecanjem veliine linije kesa do neke mere performanse se poboljsavaju,
    ali se u nekom trenutku performanse pocinju da opadaju.

    \paragraph{Asocijativnost}:
    Povecanjem stepena asocijativnost povecava se procenat pogodaka kesa.
    Potpuno asocijativni kes je u tom smislu najbolji, ali je njegova
    implementacija previse slozena, pa se zbog toga koriste set-asocijativno
    mapiranje.
    
    
    

\end{multicols}

\end{document}
